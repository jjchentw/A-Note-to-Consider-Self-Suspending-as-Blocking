\section{Introduction}

This report presents a proof to support the correctness of the schedulability test for self-suspending real-time task systems by Jane W. S. Liu in the textbook \cite{Liu:2000:RS:518501}. Since there was no proof in the book, this report provides the reasoning of the correctness.  

We define the
terminologies as follows:  A sporadic task
$\tau_i$ is released repeatedly, with each such invocation called a
job. The $j^{th}$ job of $\tau_i$, denoted $\tau_{i,j}$, is released
at time $r_{i,j}$ and has an absolute deadline at time $d_{i,j}$. Each
job of any task $\tau_i$ is assumed to have worst-case execution time $C_i$. Each job of task $\tau_i$ 
suspends for at most $S_i$ time units (across all of its suspension phases).
The response time
of a job is defined as its finishing time minus its release
time. Successive jobs of the same task are required to execute in
sequence. Associated with each task $\tau_i$ are a period (or minimum inter-arrival time) $T_i$, which
specifies the minimum time between two consecutive job releases of
$\tau_i$, and a deadline $D_i$, which specifies the relative deadline
of each such job, i.e., $d_{i,j}=r_{i,j}+D_i$. The worst-case response
time of a task $\tau_i$ is the maximum response time among all its
jobs.  The utilization of a task $\tau_i$ is defined as $U_i=C_i/T_i$.

Here, in this report, we focus on constrained-deadline task systems, in which $D_i \leq T_i$ for every task $\tau_i$. We only consider preemptive fixed-priority scheduling, in which each task is assigned with a unique priority level. We assume that the priority ordering is given.

We will focus on the analysis of task $\tau_k$. There are $k-1$ higher-priority tasks, i.e., $\tau_1, \tau_2, \ldots, \tau_{k-1}$, than task $\tau_k$.  We assume that the priority ordering is given, and the task with a smaller index has higher priority, i.e., task $\tau_i$ is with a higher-priority level than task $\tau_{i+1}$. Task $\tau_1$ is the highest-priority task in the task system. When performing schedulability analysis of task $\tau_k$, we assume that $\tau_1, \tau_2, \ldots, \tau_{k-1}$ are already verified to meet their deadlines. We also classify the $k-1$ higher-priority tasks into two sets: ${\bf T}_1$ and ${\bf T}_2$. A task $\tau_i$ is in ${\bf T}_1$ if $C_i \geq S_i$; otherwise, it is in ${\bf T}_2$. 

\section{Liu's Analysis}

To analyze the worst-case response time (or the schedulability) of task $\tau_k$, we need to in general quantify the worst-case interference of the higher-priority tasks during the execution of a job of task $\tau_k$. In the ordinary sporadic real-time tasks, i.e., $S_i=0$ for every task $\tau_i$, the so-called critical instant theorem by Liu and Layland \cite{Liu_1973} is commonly adopted. That is, the worst-case response time of task $\tau_k$ (if it is less than or equal to its period) happens when all the higher-priority tasks release a job at the same time and the subsequent jobs as early as possible (by respecting to the periods). 

The critical instant theorem does not work for self-suspending sporadic task models.  
Jane W. S. Liu \cite{Liu:2000:RS:518501} shows that the critical instant theorem can still be applied by modeling the \emph{extra delay} suffered by a task $\tau_k$ due to self-suspending behavior as a factor of blocking time, denoted as $B_k$ as follows:
\begin{itemize}
\item The blocking time contributed from task $\tau_k$ is $S_k$.
\item A higher-priority task $\tau_i$ can only block the execution of task $\tau_k$ by at most $C_i$ time units. Moreover, if $S_i < C_i$, then the blocking time contributed from task $\tau_i$ is at most $S_i$. Therefore, the contribution of a higher-priority task $\tau_i$  to $B_k$ is at most $min(C_i, S_i)$.
\end{itemize}
Therefore, 
\begin{equation}
\label{eq:Bk}
B_k = S_k + \sum_{i=1}^{k-1} min(C_i, S_i).
\end{equation}

With the above argument, a constrained-deadline task $\tau_k$ can be feasibly scheduled by the fixed-priority scheduling if
\begin{equation}
\label{eq:TDA-suspension}
\exists 0 < t \leq D_k, \qquad C_k + B_k + \sum_{i=1}^{k-1}\ceiling{\frac{t}{T_i}} C_i \leq t.
\end{equation}
  
  
  
  
  
  