\section{Introduction}

This report presents a proof to support the correctness of the feasibility test for self-suspending real-time task systems by Jane W. S. Liu in the textbook \cite{Liu:2000:RS:518501}. Since there was no proof in the book, this report provides the reasoning of the correctness.  

We define the
terminologies in this section for completeness.  A sporadic task
$\tau_i$ is released repeatedly, with each such invocation called a
job. The $j^{th}$ job of $\tau_i$, denoted $\tau_{i,j}$, is released
at time $r_{i,j}$ and has an absolute deadline at time $d_{i,j}$. Each
job of any task $\tau_i$ is assumed to have execution time $C_i$. Each job of task $\tau_i$ 
suspends for at most $S_i$ time units (across all of its suspension phases).
The response time
of a job is defined as its finishing time minus its release
time. Successive jobs of the same task are required to execute in
sequence. Associated with each task $\tau_i$ are a period $T_i$, which
specifies the minimum time between two consecutive job releases of
$\tau_i$, and a deadline $D_i$, which specifies the relative deadline
of each such job, i.e., $d_{i,j}=r_{i,j}+D_i$. The worst-case response
time of a task $\tau_i$ is the maximum response time among all its
jobs.  The utilization of a task $\tau_i$ is defined as $U_i=C_i/T_i$.

Here, in this report, we focus on constrained-deadline task systems, in which $D_i \leq T_i$ for every task $\tau_i$. We only consider preemptive fixed-priority scheduling, in which each task is assigned with a unique priority level. We assume that the priority ordering is given, and the task with a smaller index has higher priority, i.e., task $\tau_i$ is with a higher-priority level than task $\tau_{i+1}$. Task $\tau_1$ is the highest-priority task in the task system.

We will focus on the analysis of task $\tau_k$ under the assumption that $\tau_1, \tau_2, \ldots, \tau_{k-1}$ are already verified to meet their deadlines. 

\section{Liu's Analysis}

To analyze the worst-case response time of task $\tau_k$, we need to in general quantify the worst-case interference of the higher-priority tasks during the execution of a job of task $\tau_k$. In the ordinary sporadic real-time tasks, i.e., $S_i=0$ for every task $\tau_i$, the so-called critical instant theorem by Liu and Layland \cite{}

  
  
  
  
  
  