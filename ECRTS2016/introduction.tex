
The periodic/sporadic task model has been recognized as the basic
model for real-time systems with recurring executions.  
% A sporadic
% real-time task $\tau_i$ is characterized by its \emph{worst-case execution
% time} $C_i$, its \emph{minimum
%   inter-arrival time} $T_i$ and its
% \emph{relative deadline} $D_i$. A sporadic task defines an infinite
% sequence of task instances, also called \emph{jobs}, that arrive with
% the minimum inter-arrival time constraint. When a job of task $\tau_i$
% arrives at time $t$, the job should finish no later than its
% \emph{absolute deadline} $t+D_i$, and the next job of task $\tau_i$
% can only be released no earlier than $t+T_i$. For the periodic task
% model, the next job is released at time $t+T_i$. $T_i$ is
% then also referred to as the \emph{period} of $\tau_i$.
The seminal work by Liu and Layland \cite{Liu_1973} considered the
scheduling of periodic tasks and presented the schedulability analyses
based on utilization bounds to verify whether the deadlines are met or
not.  For decades, researchers in real-time systems have
devoted themselves to effective design and efficient analyses of
different recurrent task models to ensure that tasks can meet their
specified deadlines. In most of these studies, \emph{a task usually does not
 suspend itself}. That is, after a job is released, the job
is either executed or stays in the ready queue, but it is not moved to
the suspension state.  Such an assumption is valid only under the
following conditions: (1) the latency of the memory accesses and I/O
peripherals is considered to be part of the worst-case execution time
of a job, (2) there is no external device for accelerating the
computation, and (3) there is no synchronization between different
tasks on different processors in a multiprocessor or distributed
computing platform.


If a job can suspend itself before it finishes its computation,
self-suspension behaviour has to be considered. Due to the interaction
with other system components and synchronization, self-suspension
behaviour has become more visible in designing real-time embedded
systems.  Typically, the resulting suspension delays range from a few
microseconds (e.g., a write operation on a flash
drive~\cite{Kang:rtss07}) to a few hundreds of milliseconds (e.g.,
offloading computation to GPUs~\cite{Kato_2011,Liu_2014}).

There are two typical models for self-suspending sporadic task
systems: 1) the dynamic self-suspension task model, and 2) the
segmented self-suspension task model. In the \emph{dynamic}
self-suspension task model, e.g.,~\cite{ECRTS-AudsleyB04,RTAS-AudsleyB04,RTCSA-KimCPKH95,MingLiRTCSA1994,LiuChen:rtss2014,Huang_2015,ecrts16-techreport}, in addition to the worst-case execution time
$C_i$ of sporadic task $\tau_i$, we have also the worst-case
self-suspension time $S_i$ of task $\tau_i$. In the \emph{segmented} self-suspension
task model, e.g., \cite{RTCSA-BletsasA05,Kim2016,WC16-suspend-DATE,RTSS-ChenL14,Huang:multiseg,PH:rtss98}, the execution behaviour of a job of task $\tau_i$ is
specified by interleaved computation segments and self-suspension
intervals.  From the system designer's perspective, the dynamic
self-suspension model provides a simple specification by ignoring the
juncture of I/O access, computation offloading, or
synchronization. However, if the suspending behaviour can be
characterized by using a segmented pattern, the segmented
self-suspension task model can be more appropriate.

In this paper, we focus on preemptive fixed-priority scheduling for
the dynamic self-suspension task model on a uniprocessor platform. To
verify the schedulability of a given task set, this problem has been
specifically studied in
\cite{RTCSA-KimCPKH95,MingLiRTCSA1994,ECRTS-AudsleyB04,RTAS-AudsleyB04,Huang_2015}.
The recent report by Chen et al. \cite{suspension-review-jj}
% \footnote{The report, with a tentative
%   title ``Many Suspensions, Many Problems: A Review of
%   Self-Suspending Tasks in Real-Time Systems'' (all the authors in
%   this paper are contributors to this review), has been
%   finalized. However, it is under the final internal review by the
%   co-authors.  We hope the report to be filed in the early March
%   2016.}
 and the report by Bletsas et al. \cite{BletsasReport2015}
have shown that several analyses in the state-of-the-art of self-suspending tasks~\cite{ECRTS-AudsleyB04,RTAS-AudsleyB04,RTCSA-KimCPKH95,MingLiRTCSA1994}
are in fact unsafe. Unfortunately, those misconceptions propagated to several works~\cite{zeng-2011,bbb-2013,yang-2013,kim-2014,han-2014,carminati-2014,yang-2014,lakshmanan-2009}
analyzing the worst-case response time for partitioned multiprocessor
real-time locking protocols. Moreover, Liu and Chen
in \cite{LiuChen:rtss2014} provided a utilization-based schedulability
test based on a hyperbolic-form. Huang et al. \cite{Huang_2015}
explored the priority assignment under the same system model.

Furthermore, one result presented by Jane W. S. Liu in her book "Real-Time Systems"
\cite[p. 164-165]{Liu:2000:RS:518501} and implicitly used
by Rajkumar, Sha, and Lehoczky \cite[p.
267]{DBLP:conf/rtss/RajkumarSL88} for analyzing the self-suspending
behaviour due to synchronization protocols in multiprocessor systems, was never proven correct.
%However, there is no proof in
%\cite{Liu:2000:RS:518501,DBLP:conf/rtss/RajkumarSL88} to support the
%correctness of the provided schedulability tests.

\noindent\textbf{Contributions.} The contributions of this paper are
as follows:
\begin{compactitem}
\item We provide a new response analysis framework for dynamic self-suspending
  sporadic real-time tasks on a uniprocessor platform. The
  key observation is that the \emph{interference from higher-priority
    self-suspending tasks can be arbitrarily modelled as jitter or
    carry-in terms}.
\item We prove that the new analysis analytically
  dominates all the state-of-the-art results, excluding the flawed ones.  
\item We prove the
  correctness of the analysis initially proposed in \cite[p.
  164-165]{Liu:2000:RS:518501} and
  \cite[p. 267]{DBLP:conf/rtss/RajkumarSL88}, which were never proven correct in the state-of-the-art\footnote{A simplified
    version of the proof of Theorem~\ref{theorem:general-framework} to support the correctness of \cite[p. 164-165]{Liu:2000:RS:518501} and \cite[p. 267]{DBLP:conf/rtss/RajkumarSL88} is
  provided in\cite{ChenHuangNelissen}.}.
%\item We develop a few strategies to decide which higher-priority
%  tasks should be classified to associate with jitter terms and which
%  higher-priority tasks should be classified to associate with carry-in
%  terms. The methods are presented in
%  Section~\ref{sec:vector-assignment}.
\item The evaluation results presented in Section~\ref{sec:experiments} show the huge improvement (an increase of up to $50\%$ of the number of task sets that are deemed schedulable) over the state-of-the-art.
\end{compactitem}




%%% Local Variables:
%%% mode: latex
%%% TeX-master: "master.tex"
%%% End:
