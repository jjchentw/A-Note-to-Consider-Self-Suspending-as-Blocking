%*******************************************************
% Abstract
%*******************************************************
%\renewcommand{\abstractname}{Abstract}
\pdfbookmark[1]{Abstract}{Abstract}
\begingroup
\let\clearpage\relax
\let\cleardoublepage\relax
\let\cleardoublepage\relax

%\chapter*{Abstract}
 For real-time embedded systems, self-suspending behaviors can cause
  substantial performance/schedulability degradations. In this paper,
  we focus on preemptive fixed-priority scheduling for the dynamic
  self-suspension task model on uniprocessor. This
  model assumes that a job of a task can dynamically suspend itself during its execution (for instance, to wait for shared resources or access co-processors or external devices).
  The total suspension time of a job is upper-bounded, but this dynamic behavior drastically influences the interference generated by this task on lower-priority tasks. The state-of-the-art results for this task model can be classified
  into three categories (i) modeling suspension as computation, (ii)
  modeling suspension as release jitter, and (iii) modeling suspension as a blocking term.
  However, several results associated to the release jitter approach have been recently proven to be erroneous, and the concept of modeling suspension as blocking was never
  formally proven correct. This paper presents a unifying
  response time analysis framework for the dynamic self-suspending
  task model. We provide a rigorous proof and show that the existing analyses pertaining to the three categories mentioned above are analytically dominated by our proposed solution. Therefore, all those techniques are in fact correct, but they are
  inferior to the proposed response time analysis in this paper. The
  evaluation results show that our analysis framework can generate huge
  improvements (an increase of up to $50\%$ of the number of task sets
  deemed schedulable) over these state-of-the-art analyses.

\vfill

\pdfbookmark[1]{Acknowledgments}{acknowledgments}
\chapter*{Acknowledgments}
The authors would like to thank the anonymous reviewers for their suggestions and helps to improve the presentation flow and the clarity of the proof of Theorem~\ref{theorem:general-framework}. 

This paper is supported by DFG, as part of the Collaborative Research
Center SFB876 (http://sfb876.tu-dortmund.de/). This work was also
partially supported by National Funds through FCT/MEC (Portuguese
Foundation for Science and Technology) and co-financed by ERDF
(European Regional Development Fund) under the PT2020 Partnership,
within project UID/CEC/04234/2013 (CISTER); also by FCT/MEC and the EU
ARTEMIS JU within project(s) ARTEMIS/0003/2012 - JU grant nr. 333053
(CONCERTO) and ARTEMIS/0001/2013 - JU grant nr. 621429 (EMC2).

\endgroup			

\vfill
