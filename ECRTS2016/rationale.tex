\section{Rationale}
\label{sec:rationale}

Even though it can be proven that the response time analysis associated with Eq.\eqref{eq:TDA-suspension} dominates the suspension oblivious one (see Lemma~\ref{lem:dominance_oblivious} in Section~\ref{sec:dominance}), none of the analyses presented in Section~\ref{sec:existing-analyses} dominates all the others. Hence, Eqs.~\eqref{eq:TDA-jitter} and \eqref{eq:TDA-suspension} are incomparable. That is, in some cases Eq.~\eqref{eq:TDA-suspension} performs better than Eq.~\eqref{eq:TDA-jitter}, while in others Eq.~\eqref{eq:TDA-jitter} outperforms Eq.~\eqref{eq:TDA-suspension}.

\begin{example} 
\label{ex:rationale_1}  
Consider the two tasks $\tau_1 = (1, 8, 10, 10)$ and $\tau_2 = (1, 0, 15, 15)$. The worst-case response time of $\tau_1$ is obviously $9$ whatever the analysis employed. However, the upper bound on the WCRT of $\tau_2$ obtained with Eq.~\eqref{eq:TDA-jitter} is $2$, while 
%the minimum $t$ larger than $0$ such that 
%$$t=C_2 + S_2+ \ceiling{ \frac{t + R_1 - C_1}{T_1} } C_1 = 1 + 0 + \ceiling{\frac{t + 9 - 1}{10}} 1$$ 
%leading to a value of $2$.
the upper bound on the WCRT of $\tau_2$ obtained with Eq.~\eqref{eq:TDA-suspension} is $3$. %the minimum $t$ larger than $0$ such that 
%$$t=C_2 + S_2 + C_1 \ceiling{ \frac{t}{T_1} } C_1 = 1 + 0 + 1 + \ceiling{\frac{t}{10}} 1$$ 
%which results in a value of $3$.  
\end{example}

\begin{example}   
Consider the three tasks $\tau_1 = (xxx)$, $\tau_2 = (yyyy)$ and $\tau_3 = (zzz)$. With both Eqs~\eqref{eq:TDA-jitter} and~\eqref{eq:TDA-suspension}, the upper bounds computed on the WCRT of $\tau_1$ and $\tau_2$ are $x$ and $y$, respectively. 

Using Eq.~\eqref{eq:TDA-jitter}, the worst-case response time of task $\tau_3$ is upper bounded by $z$. %the minimum $t$ larger than $0$ such that 
%\begin{align*}
%t & = C_3 + S_3 +\sum_{i=1}^2 \left\lceil \frac{t + R_i - C_i}{T_i} \right\rceil C_i = 1 +\left\lceil \frac{t + 8}{10} \right\rceil 1 +\left\lceil \frac{t + 12}{15} \right\rceil 1
%\end{align*} 
%which turns to be $xxx$.
However, with Eq.~\eqref{eq:TDA-suspension}, the computed upper bound on the WCRT of $\tau_3$ turns out to be $v$. %the minimum positive $t$ such that 
%\begin{align*}
%t & = C_3 + C_2 + C_1 + \sum_{i=1}^2 \left\lceil \frac{t}{T_i} \right\rceil C_i = 12 +\left\lceil \frac{t}{10} \right\rceil 1 +\left\lceil \frac{t}{14} \right\rceil 1.
%\end{align*} 
%This equality holds when $t=xxx$. 
Therefore, in this example, the upper bound on the WCRT of $\tau_3$ computed with Eq.~\eqref{eq:TDA-suspension} is smaller than the value obtained with Eq.~\eqref{eq:TDA-jitter}, while the opposite was true in Example~\ref{ex:rationale_1}. 
\end{example}

Furthermore, there might be task sets that are indeed schedulable but that are rejected by both Eqs.~\eqref{eq:TDA-jitter} and \eqref{eq:TDA-suspension}.
 
\begin{example}  
Consider the three tasks $\tau_1 = (4, 5, 10, 10)$, $\tau_2 =(6, 1, 19, 19)$ and $\tau_3 = (4, 0, 35, 35)$. According to Eqs.~\eqref{eq:TDA-jitter} and \ref{eq:TDA-suspension}, $\tau_1$ and $\tau_2$ are both schedulable and have a respective worst-case response time of $9$ and $15$. 

For task $\tau_3$, 

Similarly, using Eq.~\eqref{eq:TDA-suspension}, the blocking term $B_3$ given by Eq.~\eqref{eq:Bk} is $4+1=5$. Therefore, the minimum $t$ larger than $0$ to satisfy $C_k + B_k +
\sum_{i=1}^{k-1}\ceiling{\frac{t}{T_i}} C_i \leq t$ is $t=37$. 

Therefore, task $\tau_3$ cannot pass the schedulability test in Eq.~(\ref{eq:TDA-suspension}). 
\end{example}


In this paper, we derive a response time analysis that draws inspiration from both Eq.~\eqref{eq:TDA-jitter} and Eq.~\eqref{eq:TDA-suspension}, combining the best of each of them. As further proven in Section~\ref{sec:dominance}, the resulting schedulability test dominates all the tests discussed in Section~\ref{sec:existing-analyses}.

