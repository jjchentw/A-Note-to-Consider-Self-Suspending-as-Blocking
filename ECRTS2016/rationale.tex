\section{Rationale}
\label{sec:rationale}

Even though it can be proven that the response time analysis associated with Eq.\eqref{eq:TDA-suspension} dominates the suspension oblivious one (see Lemma~\ref{lem:dominance_oblivious} in Section~\ref{sec:dominance}), none of the analyses presented in Section~\ref{sec:existing-analyses} dominates all the others. Hence, Eqs.~\eqref{eq:TDA-jitter} and \eqref{eq:TDA-suspension} are incomparable. That is, in some cases Eq.~\eqref{eq:TDA-suspension} performs better than Eq.~\eqref{eq:TDA-jitter}, while in others Eq.~\eqref{eq:TDA-jitter} outperforms Eq.~\eqref{eq:TDA-suspension}.

\begin{example} 
\label{ex:rationale_1}  
Consider the two tasks $\tau_1 = (4, 5, 10, 10)$ and $\tau_2 =(6, 1, 19, 19)$. The worst-case response time of $\tau_1$ is obviously $9$ whatever the analysis employed. However, the upper bound on the WCRT of $\tau_2$ obtained with Eq.~\eqref{eq:TDA-jitter} is $15$, while it is $19$ with Eq.~\eqref{eq:TDA-suspension}. The solution obtained with Eq.~\eqref{eq:TDA-jitter} is therefore tighter.

Now, let us consider one more task $\tau_3 = (4, 0, 50, 50)$. Using Eq.~\eqref{eq:TDA-jitter}, the WCRT of task $\tau_3$ is upper bounded by the smallest $t>0$ such that
$t = 4 +\ceiling{ \frac{t + 9 - 4}{10} } 4 +\ceiling{ \frac{t + 15 - 6}{19} } 6$, which turns out to be $42$. With Eq.~\eqref{eq:TDA-suspension} though, $B_3 = 4+1 = 5$ (Eq.~\eqref{eq:Bk}) and an upper bound on the WCRT of $\tau_3$ is given by the smallest $t>0$ such that 
$C_3 + B_3 + \sum_{i=1}^{2}\ceiling{\frac{t}{T_i}} C_i \leq t$. The solution to this last equation is $t=37$. Therefore, Eq.~\eqref{eq:TDA-suspension} provides a tighter bound on the WCRT of $\tau_3$ than Eq.~\eqref{eq:TDA-jitter}, while the opposite was true for $\tau_2$.
\hfill\myendproof
\end{example}

In addition to the fact that Eqs.~\eqref{eq:TDA-jitter} and \eqref{eq:TDA-suspension} are incomparable, there might be task sets for which both equations overestimate the WCRT. One such example is given below.
 
\begin{example}
\label{ex:rationale_2}  
Consider the same three tasks as in Example~\ref{ex:rationale_1}. As explained in Section~\ref{sec:jitter}, the extra interference caused by the self-suspending behavior of $\tau_1$ can be safely modeled by a release jitter equal to $R_1 - C_1 = 5$. Similarly, the extra interference caused by the self-suspension of $\tau_2$ can be modeled by a blocking time equal to $\min(C_2,S_2) = 1$ (see Section~\ref{sec:suspension-blocking}). Hence, the WCRT of $\tau_3$ is upper bounded by the smallest $t>0$ such that $t = 4 + 1 +\ceiling{ \frac{t + 5}{10} } 4 +\ceiling{ \frac{t}{19} } 6$, which turns out to be $33$. This bound on the WCRT is smaller than the estimates obtained with both Eqs.~\eqref{eq:TDA-jitter} and \eqref{eq:TDA-suspension} (see Example~\ref{ex:rationale_1}).
\hfill\myendproof
\end{example}


Example~\ref{ex:rationale_2} shows that a tighter bound on the WCRT of a task can be obtained by combining the properties of the analyses discussed in both Section~\ref{sec:jitter} and~\ref{sec:suspension-blocking}. Therefore, in this paper, we derive a response time analysis that draws inspiration from both Eqs.~\eqref{eq:TDA-jitter} and~\eqref{eq:TDA-suspension}, combining the best of each of them. As further proven in Section~\ref{sec:dominance}, the resulting schedulability test dominates all the tests discussed in Section~\ref{sec:existing-analyses}.




%%% Local Variables:
%%% mode: latex
%%% TeX-master: "master.tex"
%%% End:
