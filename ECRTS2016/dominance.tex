\section{Dominance over the State of the Art}
\label{sec:dominance}

In this section, we prove that the schedulability test presented in Corollary~\ref{corollary:general-framework} dominates all the existing tests in the state-of-the-art, in the sense that if a task set is deemed schedulable by either of the tests presented in Section~\ref{sec:existing-analyses}, then it is also deemed schedulable by Corollary~\ref{corollary:general-framework}.

\begin{Lemma}
\label{lem:dominance_oblivious}
  The schedulability test of task $\tau_k$ provided by
  Eq.~\eqref{eq:TDA-suspension} dominates that of
  Eq.~\eqref{eq:TDA-SO}.
\end{Lemma}
\begin{proof}
It is straightforward to see that
\begin{align*}
& C_k + S_k + \sum_{i=1}^{k-1}\ceiling{\frac{t}{T_i}} (C_i + S_i) \\
 \geq & C_k + S_k + \sum_{i=1}^{k-1} S_i + \sum_{i=1}^{k-1}\ceiling{\frac{t}{T_i}} C_i\\
 \geq & C_k + S_k + \sum_{i=1}^{k-1} \min(C_i, S_i) + \sum_{i=1}^{k-1}\ceiling{\frac{t}{T_i}} C_i
\end{align*}
and by using the definition of $B_k$ (i.e., Equation~\eqref{eq:Bk}), we get
\begin{align*}
C_k + S_k + \sum_{i=1}^{k-1}\ceiling{\frac{t}{T_i}} (C_i + S_i) \geq C_k + B_k + \sum_{i=1}^{k-1}\ceiling{\frac{t}{T_i}} C_i
\end{align*}
Therefore, Eq.~\eqref{eq:TDA-suspension} will always have a solution which is smaller than or equal to the solution of Eq.~\eqref{eq:TDA-SO}. This proves the lemma.
\end{proof}

\begin{Lemma}
  \label{lem:dominance_jitter}
  The schedulability test presented in
  Corollary~\ref{corollary:general-framework} dominates the
  schedulability test provided by Eq.~\eqref{eq:TDA-jitter}.
\end{Lemma}
\begin{proof}
  Consider the case where $x_1=x_2=\cdots=x_{k-1}=0$. Eq.~\eqref{eq:TDA-suspension-tighter} becomes identical to Eq.~\eqref{eq:TDA-jitter} for this particular vector assignment. Therefore, if Eq.~\eqref{eq:TDA-jitter} deems a task set as being schedulable, so does Corollary~\ref{corollary:general-framework}. This proves the lemma. 
\end{proof}
  
\begin{Lemma}
  \label{lem:dominance_blocking}
  The schedulability test presented in
  Corollary~\ref{corollary:general-framework} dominates the
  schedulability test provided by Eq.~\eqref{eq:TDA-suspension}.
\end{Lemma}
\begin{proof}
  In this proof, we first transform the worst-case response time analysis presented in Corollary~\ref{corollary:general-framework} in a more pessimistic analysis. We then prove that this more pessimistic version of Corollary~\ref{corollary:general-framework} provides the same solution than Eq.~\eqref{eq:TDA-suspension}, which then proves the lemma.
  
  Since $Q_i^{\vec{x}} \equals \sum_{j=i}^{k-1} S_j \times x_j$, it holds that $Q_i^{\vec{x}} \leq  Q_1^{\vec{x}}$ for $i=1,2,\ldots,k-1$. It follows that
  \begin{align*}
  & C_k + S_k + \sum_{i=1}^{k-1}\ceiling{\frac{t+Q_i^{\vec{x}}+(1-x_i)(R_i-C_i)}{T_i}} \\
  \stackrel{(Q_i^{\vec{x}} \leq  Q_1^{\vec{x}})}{\leq} & C_k + S_k + \sum_{i=1}^{k-1}\ceiling{\frac{t+Q_1^{\vec{x}}+(1-x_i)(R_i-C_i)}{T_i}} \\
  \stackrel{(R_i \leq D_i \leq T_i)}{\leq} & C_k + S_k + \sum_{i=1}^{k-1}\ceiling{\frac{t+Q_1^{\vec{x}}+(1-x_i)T_i}{T_i}} \\
  \stackrel{(x_i \in \{0,1\})}{=} & C_k + S_k + \sum_{i=1}^{k-1} (1-x_i)C_i + \sum_{i=1}^{k-1}\ceiling{\frac{t+Q_1^{\vec{x}}}{T_i}} \\
  \end{align*}
  
  Therefore, the smallest positive value $t$ such that  
  \begin{equation}
  \label{eq:proof_dominance_blocking}
  C_k + S_k + \sum_{i=1}^{k-1} (1-x_i)C_i + \sum_{i=1}^{k-1}\ceiling{\frac{t+Q_1^{\vec{x}}}{T_i}} \leq t
  \end{equation}
  is always larger than or equal to the solution of Eq.~\eqref{eq:TDA-suspension-tighter}. 
  
  Subtituting $(t+Q_1^{\vec{x}})$ by $\theta$ in Eq.~\eqref{eq:proof_dominance_blocking}, we get that $R_k$ is upper bounded by the minimum value $(\theta-Q_1^{\vec{x}})$ greater than $0$ (and therefore by the smallest $\theta > 0$) such that 
  \begin{align}
  & C_k + S_k + \sum_{i=1}^{k-1} (1-x_i)C_i + \sum_{i=1}^{k-1}\ceiling{\frac{\theta}{T_i}}\leq \theta-Q_1^{\vec{x}} \nonumber \\
  \Leftrightarrow \;\; & C_k + S_k + Q_1^{\vec{x}} + \sum_{i=1}^{k-1} (1-x_i)C_i + \sum_{i=1}^{k-1}\ceiling{\frac{\theta}{T_i}}\leq \theta \nonumber \\
\Leftrightarrow \;\; & C_k + S_k + \sum_{i=1}^{k-1}(x_i S_i + (1-x_i) C_i) + \sum_{i=1}^{k-1}\ceiling{\frac{\theta}{T_i}}C_i \leq \theta \label{eq:proof_dominance_blocking_final}.
    \end{align}
    
    
    Now, consider the particular vector assignment $\vec{x}$ in which 
  \begin{equation*}
    x_i =
    \begin{cases}
      1 &\mbox{if } S_i \leq    C_i\\
      0 & \mbox{otherwise},
    \end{cases}
  \end{equation*}
  for $i=1,2,\ldots,k-1$. 
    By the definition of $B_k$ (i.e., Eq.~\eqref{eq:Bk}), we get that 
    $$B_k = S_k + \sum_{i=1}^{k-1} \min(C_i, S_i) = S_k + \sum_{i=1}^{k-1} \left(x_i
    S_i + (1-x_i) C_i\right)$$
Eq.~\eqref{eq:proof_dominance_blocking_final} thus becomes identical to Eq.~\eqref{eq:TDA-suspension}. Therefore, if Eq.~\eqref{eq:TDA-suspension} deems a task set as being schedulable, so does Corollary~\ref{corollary:general-framework}. 
\end{proof}

\begin{theorem}
  \label{theorem:dominance}
  The schedulability test presented in Corollary~\ref{corollary:general-framework} dominates the schedulability tests provided by Equations~\eqref{eq:TDA-SO}, \eqref{eq:TDA-jitter}, and~\eqref{eq:TDA-suspension}.
\end{theorem}
\begin{proof}
It ias a direct application of Lemmas~\ref{lem:dominance_oblivious}, \ref{lem:dominance_jitter} and~\ref{lem:dominance_blocking}.
\end{proof}


As a corollary of this theorem, it directly follows that all the response time analyses discussed in Section~\ref{sec:existing-analyses} are in fact correct. This provides the first proof of correctness for Eq.~\eqref{eq:TDA-suspension}, which was initially presented in \cite{Liu:2000:RS:518501} but never proven correct.

\begin{theorem}
  \label{theorem:correctness_soa}
  The schedulability tests provided by Eqs~\eqref{eq:TDA-SO}, \eqref{eq:TDA-jitter}, and~\eqref{eq:TDA-suspension} are all correct.
\end{theorem}
\begin{proof}
It directly results from the two following facts,
\begin{itemize}[leftmargin=0]
\item[(i)] by Theorem~\ref{theorem:dominance}, the schedulability test presented in Corollary~\ref{corollary:general-framework} dominates the schedulability tests provided by Equations~\eqref{eq:TDA-SO}, \eqref{eq:TDA-jitter}, and~\eqref{eq:TDA-suspension};
\item [(ii)] as proven in Section~\ref{sec:proof-th1}, Corollary~\ref{corollary:general-framework} is correct.
\end{itemize}
\end{proof}
