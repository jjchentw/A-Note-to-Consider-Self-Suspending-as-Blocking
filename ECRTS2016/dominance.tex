
In this section, we prove that the schedulability test presented in Corollary~\ref{corollary:general-framework} dominates all the existing tests in the state-of-the-art, in the sense that if a task set is deemed schedulable by either of the tests presented in Section~\ref{sec:existing-analyses}, then it is also deemed schedulable by Corollary~\ref{corollary:general-framework}.

\begin{Lemma}
\label{lem:dominance_oblivious}
  The schedulability test of task $\tau_k$ provided by
  Eq.~\eqref{eq:TDA-suspension} dominates that of
  Eq.~\eqref{eq:TDA-SO}.
\end{Lemma}
\begin{proof}
For any $t > 0$, it is straightforward to see that
\begin{align*}
& C_k + S_k + \sum_{i=1}^{k-1}\ceiling{\frac{t}{T_i}} (C_i + S_i) \\
 \geq & C_k + S_k + \sum_{i=1}^{k-1} S_i + \sum_{i=1}^{k-1}\ceiling{\frac{t}{T_i}} C_i\\
 \geq & C_k + S_k + \sum_{i=1}^{k-1} \min(C_i, S_i) + \sum_{i=1}^{k-1}\ceiling{\frac{t}{T_i}} C_i
\end{align*}
and by using the definition of $B_k$ (i.e., in Section~\ref{sec:suspension-blocking}), we get
\begin{align*}
C_k + S_k + \sum_{i=1}^{k-1}\ceiling{\frac{t}{T_i}} (C_i + S_i) \geq C_k + B_k + \sum_{i=1}^{k-1}\ceiling{\frac{t}{T_i}} C_i
\end{align*}
Therefore, Eq.~\eqref{eq:TDA-suspension} will always have a solution which is smaller than or equal to the solution of Eq.~\eqref{eq:TDA-SO}. This proves the lemma.
\end{proof}

\begin{Lemma}
  \label{lem:dominance_jitter}
  The schedulability test presented in
  Corollary~\ref{corollary:general-framework} dominates the
  schedulability test provided by Eq.~\eqref{eq:TDA-jitter}.
\end{Lemma}
\begin{proof}
  Consider the case where $x_1=x_2=\cdots=x_{k-1}=0$. Eq.~\eqref{eq:TDA-suspension-tighter} becomes identical to Eq.~\eqref{eq:TDA-jitter} for this particular vector assignment. Therefore, if Eq.~\eqref{eq:TDA-jitter} deems a task set as being schedulable, so does Corollary~\ref{corollary:general-framework}. This proves the lemma. 
\end{proof}
  
\begin{Lemma}
  \label{lem:dominance_blocking}
  The schedulability test presented in
  Corollary~\ref{corollary:general-framework} dominates the
  schedulability test provided by Eq.~\eqref{eq:TDA-suspension}.
\end{Lemma}
\begin{proof}
  In this proof, we first transform the worst-case response time analysis presented in Corollary~\ref{corollary:general-framework} in a more pessimistic analysis. We then prove that this more pessimistic version of Corollary~\ref{corollary:general-framework} provides the same solution as Eq.~\eqref{eq:TDA-suspension}, which then proves the lemma. Due to space limitation, the proof is in Appendix\citetechreport{}.
\end{proof}

\begin{theorem}
  \label{theorem:dominance}
  The schedulability test presented in Corollary~\ref{corollary:general-framework} dominates the schedulability tests provided by Equations~\eqref{eq:TDA-SO}, \eqref{eq:TDA-jitter}, and~\eqref{eq:TDA-suspension}.
\end{theorem}
\begin{proof}
It is a direct application of Lemmas~\ref{lem:dominance_oblivious}, \ref{lem:dominance_jitter} and~\ref{lem:dominance_blocking}.
\end{proof}


As a corollary of this theorem, it directly follows that all the response time analyses discussed in Section~\ref{sec:existing-analyses} are in fact correct. This provides the first proof of correctness for Eq.~\eqref{eq:TDA-suspension}, which was initially presented in \cite{Liu:2000:RS:518501} but never proven correct.

\begin{theorem}
  \label{theorem:correctness_soa}
  The schedulability tests provided by Eqs~\eqref{eq:TDA-SO}, \eqref{eq:TDA-jitter}, and~\eqref{eq:TDA-suspension} are all correct.
\end{theorem}
\begin{proof}
It directly results from the two following facts,
\begin{compactitem}
\item[(i)] by Theorem~\ref{theorem:dominance}, the schedulability test presented in Corollary~\ref{corollary:general-framework} dominates the schedulability tests provided by Equations~\eqref{eq:TDA-SO}, \eqref{eq:TDA-jitter}, and~\eqref{eq:TDA-suspension};
\item [(ii)] as proven in Section~\ref{sec:proof-th1}, Corollary~\ref{corollary:general-framework} is correct.
\end{compactitem}
\end{proof}




%%% Local Variables:
%%% mode: latex
%%% TeX-master: "master.tex"
%%% End:
