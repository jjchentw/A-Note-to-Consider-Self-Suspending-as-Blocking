To analyze the worst-case response time (or the schedulability) of a task $\tau_k$, one usually needs to quantify the worst-case interference exerted by the higher-priority tasks on the execution of any job of task $\tau_k$. In the ordinary sequential sporadic real-time task model, i.e., when $S_i=0$ for every task $\tau_i$, the so-called critical instant theorem by Liu and Layland \cite{Liu_1973} is commonly adopted. That is, the worst-case response time of task $\tau_k$ (if it is less than or equal to its period) happens for the first job of task $\tau_k$ when (i) $\tau_k$ and all the higher-priority tasks release their first jobs synchronously and (ii) all their subsequent jobs are released as early as possible (i.e., with a rate equal to their periods).  However,  this definition of the
critical instant does not hold for self-suspending tasks.  


The analysis of self-suspending task systems requires to model the self-suspending behavior of both the task $\tau_k$ under analysis and the higher priority tasks that interfere with $\tau_k$. The techniques employed to model the self-suspension are usually different for $\tau_k$ and the higher priority tasks. The worst-case for $\tau_k$ happens when its jobs suspend whenever there is no
higher-priority job in the system. The resulting behavior is therefore similar as if the
suspension time $S_k$ of task $\tau_k$ was converted
into computation time (see \cite{Huang_2015} for more detailed explanations). 
Second, for the higher-priority tasks,
we need to consider the self-suspension behaviour that may result in
the largest possible interference for task $\tau_k$.
%With the above concept in mind, 
There exist three approaches in the state-of-the-art 
that are potentially sound to perform the schedulability analysis of self-suspending tasks:
\begin{compactitem}
\item modeling the suspension as execution, also known as the suspension-oblivious analysis (see Section~\ref{sec:suspension-oblivious});
\item modeling the suspension as release jitter (see Section~\ref{sec:jitter});
\item modeling the suspension as blocking time (see Section~\ref{sec:suspension-blocking}).
\end{compactitem}
We later prove in Section~\ref{sec:dominance} that all these approaches are analytically correct. 

\subsection{Suspension-Oblivious Analysis}
\label{sec:suspension-oblivious} 
The simplest analysis consists in converting the suspension time $S_i$ of each %higher-priority 
task $\tau_i$ as a part of its computation
time. Therefore, a constrained-deadline task $\tau_k$ can be feasibly
scheduled by a fixed-priority scheduling algorithm if
\begin{equation}
\label{eq:TDA-SO}
\exists t \mid 0 < t \leq D_k, \quad C_k + S_k + \sum_{i=1}^{k-1}\ceiling{\frac{t}{T_i}} (C_i+S_i) \leq t.
\end{equation}

\mysection{Modeling the Suspension as Release Jitter}
\label{sec:jitter}

Another approach consists in modeling the impact of the self-suspension $S_i$ of each higher priority task $\tau_i$ as release jitter. Several works in the state-of-the-art \cite{ECRTS-AudsleyB04,RTAS-AudsleyB04,RTCSA-KimCPKH95,MingLiRTCSA1994} upper bounded the release jitter with $S_i$. However, it has been recently shown in~\cite{BletsasReport2015} that this upper bound is unsafe and  the release jitter of task $\tau_i$ can in fact be larger than $S_i$. 

Nevertheless, it was proven in the same document \cite{BletsasReport2015} that the jitter of a higher-priority task $\tau_i$ can be safely upper bounded by $R_i-C_i$. It results that a task
$\tau_k$ with a constrained deadline can be feasibly scheduled under fixed-priority if
\begin{equation}
\label{eq:TDA-jitter}
\exists t \mid 0 < t \leq D_k, \quad C_k + S_k + \sum_{i=1}^{k-1}\ceiling{\frac{t+R_i-C_i}{T_i}} C_i \leq t.
\end{equation}

\mysection{Modeling the Suspension as Blocking Time}
\label{sec:suspension-blocking}

In \cite[p. 164-165]{Liu:2000:RS:518501}, Liu proposed a solution to study the schedulability of a self-suspending task $\tau_k$ by modeling the extra delay suffered by $\tau_k$ due to the self-suspension behavior of each task in $\tau$ as a blocking time.\footnote{Even though the authors in this paper are able to provide a proof to support the correctness, the authors are not able to provide any rationale behind this method which treats suspension time as blocking time.} This blocking time has been defined as follows:
\begin{itemize}
\item The blocking time contributed from task $\tau_k$ is $S_k$. 
\item A higher-priority task $\tau_i$ can block the execution of task $\tau_k$ for at most $\min(C_i, S_i)$ time units.
\end{itemize}
%As a result, if $S_i < C_i$, then the blocking time contributed from task $\tau_i$ is at most $S_i$. Therefore, the contribution of a higher-priority task $\tau_i$  to $B_k$ is at most $b_i=min(C_i, S_i)$.
An upper bound on the blocking time is therefore given by:
$B_k = S_k + \sum_{i=1}^{k-1} \min(C_i, S_i).$
In \cite{Liu:2000:RS:518501}, the blocking time is then used to derive a utilization-based schedulability test for rate-monotonic scheduling. Namely, it is stated that, if $T_i=D_i$ for every task $\tau_i \in \tau$ and $\frac{C_k+B_k}{T_k} + \sum_{i=1}^{k-1} U_i \leq k (2^{\frac{1}{k}}-1)$, then $\tau_k$ can be feasibly scheduled with rate-monotonic scheduling. 
  

The same concept was also implicitly used by Rajkumar, Sha, and Lehoczky in~\cite[p. 267]{DBLP:conf/rtss/RajkumarSL88} for analyzing the impact of the self-suspension of a task due to the utilization of synchronization protocols in multiprocessor systems. (See Appendix\citetechreport{} for details.)
If the above argument is correct, we can further prove that a constrained-deadline task $\tau_k$ can be feasibly scheduled under fixed-priority scheduling if
\begin{equation}
\label{eq:TDA-suspension}
\exists t \mid 0 < t \leq D_k, \quad C_k + B_k + \sum_{i=1}^{k-1}\ceiling{\frac{t}{T_i}} C_i \leq t.
\end{equation}
However, there is no proof in
\cite{Liu:2000:RS:518501} nor in \cite{DBLP:conf/rtss/RajkumarSL88} to support the correctness of those tests. Therefore, in Section~\ref{sec:dominance}, we provide a proof (see Theorem~\ref{theorem:correctness_soa}) of the correctness of Equation~\eqref{eq:TDA-suspension}.

%It is not difficult to see that
%the test in Eq.~\eqref{eq:TDA-suspension} dominates that in
%Eq.~\eqref{eq:TDA-SO}.
%
%\begin{Lemma}
%  The schedulability test of task $\tau_k$ provided by
%  Eq.~\eqref{eq:TDA-suspension} dominates that of
%  Eq.~\eqref{eq:TDA-SO}.
%\end{Lemma}
%\begin{proof}
%  This is rather trivial, and the proof is omitted.
%\end{proof}




%%% Local Variables:
%%% mode: latex
%%% TeX-master: "master.tex"
%%% End:
