\section{A Unifying Analysis Framework}
\label{sec:analysis}


 \begin{table*}[t]
    \centering
    \renewcommand{\arraystretch}{1.7}
\scalebox{0.95}{
    \begin{tabular}{|c|c|c|c|c|}
\hline
    $\vec{x}$ & Case 1: $(0, 0)$ & Case 2: $(0, 1)$ & Case 3: $(1, 0)$ & Case 4:$(1, 1)$\\
\hline
    $(Q_1^{\vec{x}}, Q_2^{\vec{x}})$ & $(0, 0)$ & $(1, 1)$ & $(5, 0)$ & $(6, 1)$\\
\hline
    condition of Eq.~\eqref{eq:TDA-suspension-tighter} & $4+ \ceiling{\frac{t+0+5}{10}} 4 + \ceiling{\frac{t+0+9}{19}} 6 \leq t$ & $4+ \ceiling{\frac{t+1+5}{10}} 4 + \ceiling{\frac{t+1+0}{19}} 6 \leq t$ & $4+ \ceiling{\frac{t+5+0}{10}} 4 + \ceiling{\frac{t+0+9}{19}} 6 \leq t$ & $4+ \ceiling{\frac{t+6+0}{10}} 4 + \ceiling{\frac{t+1+0}{19}}6 \leq t$\\      
\hline
upper bound of $R_3$ & $42$ & $32$ & $42$ & $32$\\
\hline
    \end{tabular}}
    \caption{Detailed procedure in Example~\ref{ex:general_framework} for deriving the upper bound of $R_3$, with $R_1-C_1=5$ and $R_2-C_2=9$.}
    \label{tab:example3-calculation}
  \end{table*}

%\subsection{A new Response time Analysis}

In all this section, we implicitly assume that $R_i
\leq D_i, \forall \tau_i \mid 1 \leq i \leq k-1$.  This assumption is implicitly used as a fact in all the theorems and lemmas. \emph{Therefore, the worst-case response time or the schedulability of task $\tau_k$ has to be verified from $k=1,2,\ldots, n$. Here we only focus on the analysis of a certain task $\tau_k$, under the assumption that we have already validated that $R_i
\leq D_i \leq T_i, \forall \tau_i \mid 1 \leq i \leq k-1$ (by using any method in this section or Section~\ref{sec:existing-analyses}). } 
Our key result in
this paper are the two following theorems:

\begin{theorem}
   \label{theorem:general-framework}
   Suppose that all tasks $\tau_\ell \mid 1 \leq \ell \leq k$ are schedulable, (i.e., $R_\ell \leq T_\ell$). Then, for any arbitrary vector assignment $\vec{x} = (x_1, x_2, \ldots, x_{k-1})$, in which $x_i$ is either $0$ or $1$, the worst-case response time $R_k$ of $\tau_k$ is upper bounded by the minimum $t$ larger than $0$ such that 
   {\small \begin{equation} \label{eq:TDA-suspension-tighter0} 
       C_k + S_k + \sum_{i=1}^{k-1}\ceiling{\frac{t+Q_i^{\vec{x}}+(1-x_i)(R_i-C_i)}{T_i}} C_i \leq t
     \end{equation}}where $Q_i^{\vec{x}} \equals \sum_{j=i}^{k-1} (S_j \times x_j)$.
 \end{theorem} 
\begin{theorem}
   \label{theorem:general-framework-not-feasible}
   Suppose that $\tau_k$ is not schedulable (i.e., $R_k > T_k$). For any arbitrary vector assignment
   $\vec{x} = (x_1, x_2, \ldots, x_{k-1})$, in which $x_i$ is either
   $0$ or $1$,  we have $\forall t | 0 < t \leq T_k$,
   {\small \begin{equation*}
       C_k + S_k + \sum_{i=1}^{k-1}\ceiling{\frac{t+Q_i^{\vec{x}}+(1-x_i)(R_i-C_i)}{T_i}} C_i > t
     \end{equation*}}where $Q_i^{\vec{x}} \equals \sum_{j=i}^{k-1} (S_j \times x_j)$.
 \end{theorem} 
 %We will explain the resulting properties from
 %Theorem~\ref{theorem:general-framework} first, by leaving the proof
 %to Section~\ref{sec:proof-th1} since it is pretty long.   
By Theorems~\ref{theorem:general-framework} and~\ref{theorem:general-framework-not-feasible}, we can directly derive the
following schedulability test.

 \begin{Corollary}
   \label{corollary:general-framework}
   If $\forall \tau_i \mid 1 \leq i < k, ~R_i \leq T_i$, and if there is a vector $\vec{x} = (x_1, x_2, \ldots,
   x_{k-1})$ with $x_i \in \{ 0, 1\}$, such that 
   {\small \begin{align} 
   \label{eq:TDA-suspension-tighter} 
       & \exists t | 0 < t \leq D_k,  \nonumber \\
       & C_k + S_k + \sum_{i=1}^{k-1}\ceiling{\frac{t+Q_i^{\vec{x}}+(1-x_i)(R_i-C_i)}{T_i}} C_i \leq t
     \end{align}}
     where $Q_i^{\vec{x}} \equals \sum_{j=i}^{k-1} (S_j \times x_j)$, then the constrained-deadline task $\tau_k$ is schedulable under fixed-priority.
 \end{Corollary}
 \begin{proof}
Let $t^*$ be the first positive value of $t$ respecting Eq. \eqref{eq:TDA-suspension-tighter}. By the assumptions stated in the claim, $t^*$ exists and $t^*$ is smaller than or equal to the deadline $D_k$. By Theorems~\ref{theorem:general-framework} and~\ref{theorem:general-framework-not-feasible}, $t^*$ exists and is smaller than or equal to $D_k$ only if $\tau_k$ is schedulable.
 \end{proof}

 
  The proof of correctness of Theorems~\ref{theorem:general-framework} and~\ref{theorem:general-framework-not-feasible}, and hence Corollary~\ref{corollary:general-framework} is provided in Section~\ref{sec:proof-th1}. Moreover, we will later prove in Section~\ref{sec:dominance}, that Corollary~\ref{corollary:general-framework} in fact dominates all the analyses discussed in Section~\ref{sec:existing-analyses}.
 
 We now use the same example as in Section~\ref{sec:rationale}, to demonstrate how
Corollary~\ref{corollary:general-framework} can be applied.
 
 \begin{example}
 \label{ex:general_framework}
  Consider the same three tasks used in Examples~\ref{ex:rationale_1} and~\ref{ex:rationale_2}, i.e., $\tau_1 = (4, 5, 10, 10)$, $\tau_2 =(6, 1, 19, 19)$ and $\tau_3 = (4, 0, 50, 50)$. By the analysis in Example~\ref{ex:rationale_1}, $R_1$ is upper bounded by $9$ and $R_2$ is upper bounded by $15$. Let us assume $R_1=9$ and $R_2=15$ in the rest of this example. There are four possible vector assignments $\vec{x}$ when considering the schedulability of task $\tau_3$ with Corollary~\ref{corollary:general-framework}.
% \noindent\textbf{Case 1.} $\vec{x} = (0 , 0)$: In this case, Theorem~\ref{theorem:general-framework} states that $R_k$ is upper bounded by the minimum $t$ under $0 < t \leq D_3$ such that $4+ \ceiling{\frac{t+5}{10}} 4 + \ceiling{\frac{t+9}{19}} 6 \leq t$. Note that this equation is identical to the schedulability test discussed in Section~\ref{sec:jitter}, and hence, as shown in Example~\ref{ex:rationale_1}, we get that $R_k \leq 42$.
%
% \noindent\textbf{Case 2.} $\vec{x} = (0 , 1)$:
%  In this case, Theorem~\ref{theorem:general-framework} states that $R_k$ is upper bounded by  the minimum $t$
% under $0 < t \leq D_3$ that satisfies $4+ \ceiling{\frac{t+6}{10}} 4 + \ceiling{\frac{t+1}{19}} 6 \leq t$. As a solution, we get that $R_k \leq 32$. % due to $4+ \ceiling{\frac{32+7}{10}} 4 + \ceiling{\frac{32+1}{19}} 6=32$.
%       
% \noindent\textbf{Case 3.} $\vec{x} = (1 , 0)$:
%  In this case, we look for the minimum $t$ such that $4+ \ceiling{\frac{t+5}{10}}\cdot 4 + \ceiling{\frac{t+9}{19}}\cdot 6 \leq t$. Hence, we get $R_k \leq 42$.
%
% \noindent\textbf{Case 4.} $\vec{x} = (1 , 1)$:
%  In this case, Theorem~\ref{theorem:general-framework} states that $R_k$ is upper bounded by  the minimum $t$
% under $0 < t \leq T_3$ such that $
%        4+ \ceiling{\frac{t+6}{10}}\cdot 4 + \ceiling{\frac{t+1}{19}}\cdot 6 \leq t$ leading to $R_k \leq 32$.% due to $4+ \ceiling{\frac{32+6}{10}} 4 + \ceiling{\frac{32+1}{19}} 6=32$.
%
% Among the above four cases, the tests in Cases 2 and 4 are the tightest. Therefore, by
% Corollary~\ref{corollary:general-framework}, $\tau_3$ is
% schedulable under fixed-priority.
%
The corresponding procedure to use these four vector assignments can be found in Table~\ref{tab:example3-calculation}. 
Among the above four cases, the tests in Cases 2 and 4 are the
tightest. 
% Therefore, by
%  Corollary~\ref{corollary:general-framework}, $\tau_3$ is
%  schedulable under fixed-priority.
\hfill\myendproof
 \end{example}

Note also that the upper bound on $R_3$ computed in
Example~\ref{ex:general_framework}, is lower than the estimated worst-case response time obtained in Example~\ref{ex:rationale_2}. The response time analysis presented in Corollary~\ref{corollary:general-framework} is therefore tighter than the simple combination of existing analysis techniques as proposed in Example~\ref{ex:rationale_2}.



\section{Proof of Liu's Analysis}  

This section provides the proof to support the correctness of the test in Eq. \eqref{eq:TDA-suspension}. First, it should be easy to see that we can convert the suspension time of task $\tau_k$ into computation. This has been done by many researchers, e.g., the proof in Lemma 3 in the paper by Liu and Chen \cite{Liu_2014}. The remaining part is to show that the additional interference due to self-suspension from a higher-priority task $\tau_i$ is at most $min(C_i, S_i)$. The interference to be at most $C_i$ has been provided in the literature as well, e.g., \cite{Rajkumar_1990}\cite{Liu_2014}. However, the argument about blocking task $\tau_k$ due to a higher-priority task $\tau_i$ by at most $S_i$ amount of time is not very clear. 

We can prove the correctness of Eq. \eqref{eq:TDA-suspension} by using a similar proof of the critical instant theorem of the ordinary sporadic task system. From the above discussions, we can greedily convert the suspension time of task $\tau_k$ to its computation time. For notational brevity, let $C_k'$ be $C_k + S_k$. We call this converted version of task $\tau_k$ as task $\tau_k'$. Our analysis is also based on a very simple observation as follows:
\begin{lemma}
\label{lemma:remove-lower-priority}
  For a schedule, based on preemptive fixed-priority scheduling, removing a lower-priority job arrived at time $t$ does not change the schedule for executing the higher-priority jobs after time $t$.
\end{lemma}
\begin{proof}
  This is due to the preemptive scheduling. The removal of the lower-priority job has no impact at all on the higher-priority jobs.
\end{proof}
\begin{lemma}
\label{lemma:remove-same-task}
  For a schedule, based on preemptive fixed-priority scheduling, if the worst-case response time of task $\tau_i$ is no more than its period $T_i$, removing a job of task $\tau_i$ arriving at time $t$ does not change the schedule for the remaining jobs of task $\tau_i$.
\end{lemma}
\begin{proof}
  The removal of the job of task $\tau_i$ has no impact on the higher-priority jobs as in Lemma \ref{lemma:remove-lower-priority}. Since the worst-case response time of task $\tau_i$ is no more than the period, the execution of the other jobs of task $\tau_i$ is also not affected by the removal of the job.
\end{proof}

Let $R_k'$ be the minimum $t > 0$ such that  $C_k + B_k + \sum_{i=1}^{k-1}\ceiling{\frac{t}{T_i}} C_i = t$, i.e., Eq. \ref{eq:TDA-suspension} holds. The following lemma shows that $R_k'$ is a safe upper bound if the worst-case response time of task $\tau_k'$ is no more than $T_k$.

\begin{lemma}
\label{lemma:critical}
 $R_k'$ defined is a safe upper bound of the worst-case response time of task $\tau_k'$ in the self-suspending task system if its worst-case response time is no more than $T_k$.
\end{lemma}
\begin{proof}
According to the above definitions, we only need to show that $R_k'$ is a safe upper bound of the worst-case response time of task $\tau_k'$ by converting the suspension time of task $\tau_k$ as computation. We consider a given schedule in the task system with $\tau_1, \tau_2, \ldots, \tau_{k-1}, \tau_k', \tau_{k+1}, \ldots$. Since we consider fixed-priority preemptive scheduling, we can safely remove all the lower priority tasks without changing any execution pattern of the higher-priority tasks by Lemma \ref{lemma:remove-lower-priority}. Moreover, since we assume that the worst-case response time of task $\tau_k'$ is no more than $T_k$, there is no impact on the schedule of a job of task $\tau_k'$ if all the other jobs of task $\tau_k'$ are removed, as shown in Lemma \ref{lemma:remove-same-task}. 


Therefore, for the rest of the proof, we only have to analyze the response time of a job of task $\tau_k'$ released at time $z$, in which all the other jobs of task $\tau_k'$ and other lower priority jobs are removed. Suppose that this job of task $\tau_k$ has a response time $\rho$ in the above schedule. This means that the system is busy for executing the higher-priority tasks or the job of task $\tau_k'$ from $z$ to $z+\rho$. In the above schedule, let $t_{k}$ be the latest moment before $z$ such that the processor does not run any job. That is, from $t_k$ to $z$, certain higher-priority tasks are executed by the processor. Apparently, we can change the release time of the job of task $\tau_k'$ to $t_k$. The response time of the job becomes $\rho+z-t_k \geq \rho$. 

Up to here, the proof is very similar to the proof of the critical instant theorem of the ordinary sporadic real-time task systems. However, for self-suspending task systems, we need to consider that a job of task $\tau_i$ suspends itself before $t_k$ and resumes after $t_k$.  Fortunately, each higher-priority task only has such a so-called \emph{carry-in} job due to the assumption that the higher-priority tasks can finish before their periods. However, analyzing the workload of such carry-in jobs due to self-suspension is non-trivial. One can conclude that each job of task $\tau_i$ has execution time up to $C_i$. This is fine with $S_i \geq C_i$. If $S_i < C_i$, we explain how to further extend the analysis window further iteratively. 


In each iteration, we will define $t_j$ as the \emph{release time} of the first job of task $\tau_j$, starting from $j=k-1, k-2, \ldots, 1$, in the revised schedule. Let $y$ be the release time of the job (arrived before $t_{j+1}$) of task $\tau_j$ that has not yet finished at time $t_{j+1}$. There are a few cases:
\begin{itemize}
\item There is no such a job of task $\tau_j$: Removing all the jobs of task $\tau_j$ arrived before $t_{j+1}$ has no impact on the schedule of the higher-priority jobs executed after $t_{j+1}$ by Lemma \ref{lemma:remove-lower-priority}. Therefore, we simply set $t_j$ to $t_{j+1}$ and remove all the jobs of task $\tau_j$ arriving before $t_{j+1}$ in the schedule
\item There is such a job of task $\tau_j$ with $y < t_{j+1}$:  Removing all the jobs of task $\tau_j$ arrived before $y$ has no impact on the schedule of the higher-priority jobs executed after $t_{j+1}$ by Lemma \ref{lemma:remove-lower-priority}. Therefore, we remove all the jobs of task $\tau_j$ arrived before $y$ in the schedule. There are two subcases:
\begin{itemize}
\item If task $\tau_j$ is in ${\bf T}_1$, i.e., $S_j < C_j$: For such a case, we set $t_{j}$ to $y$. Moreover, we also know that the maximum idle time of the system is at most $S_j$ from $t_j$ to $t_{j+1}$
\item If task $\tau_j$ is in ${\bf T}_2$, i.e., $S_j \geq C_j$: For such a case, we set $t_{j}$ to $t_{j+1}$. Let $C_j'$ be the remaining execution time for the job of task $\tau_j$, unfinished at time $t_j$. We know that $C_j'$ is at most $C_j$. Here, we remove the job of task $\tau_j$ arrived at time $y$ and release a new job with execution time $C_j'$  at time $t_j$ with the same priority level of task $\tau_j$. Clearly, this has no impact on the execution of the higher-priority jobs executed after $t_j$.
\end{itemize}

The above construction of $t_{k-1}, t_{k-2}, \ldots, t_1$ is well-defined. We know that the maximum idle time of the above schedule due to self-suspension from $t_1$ to $t_k$ is at most $\sum_{\tau_i \in {\bf T}_1} S_i$. 

\end{itemize}
 
\end{proof}
  
  

  
  
  
  
  
  
  
  
  
  
  
  
  
  
  
  
  
  
  
  
  
  
  
  
  



%%% Local Variables:
%%% mode: latex
%%% TeX-master: "master.tex"
%%% End:
