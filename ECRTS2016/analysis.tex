

\ifbool{tutechreport}{
\def\tableAscale{0.55}
}{
\def\tableAscale{0.95}
}


 \begin{table*}[t]
    \centering
    \renewcommand{\arraystretch}{1.7}
\scalebox{\tableAscale}{
    \begin{tabular}{|c|c|c|c|c|}
\hline
    $\vec{x}$ & Case 1: $(0, 0)$ & Case 2: $(0, 1)$ & Case 3: $(1, 0)$ & Case 4:$(1, 1)$\\
\hline
    $(Q_1^{\vec{x}}, Q_2^{\vec{x}})$ & $(0, 0)$ & $(1, 1)$ & $(5, 0)$ & $(6, 1)$\\
\hline
    condition of Eq.~\eqref{eq:TDA-suspension-tighter} & $4+ \ceiling{\frac{t+0+5}{10}} 4 + \ceiling{\frac{t+0+9}{19}} 6 \leq t$ & $4+ \ceiling{\frac{t+1+5}{10}} 4 + \ceiling{\frac{t+1+0}{19}} 6 \leq t$ & $4+ \ceiling{\frac{t+5+0}{10}} 4 + \ceiling{\frac{t+0+9}{19}} 6 \leq t$ & $4+ \ceiling{\frac{t+6+0}{10}} 4 + \ceiling{\frac{t+1+0}{19}}6 \leq t$\\      
\hline
upper bound of $R_3$ & $42$ & $32$ & $42$ & $32$\\
\hline
    \end{tabular}}
    \caption{Detailed procedure in Example~\ref{ex:general_framework} for deriving the upper bound of $R_3$, with $R_1-C_1=5$ and $R_2-C_2=9$.}
    \label{tab:example3-calculation}
  \end{table*}

%\subsection{A new Response time Analysis}

In all this section, we implicitly assume that $R_i
\leq D_i, \forall \tau_i \mid 1 \leq i \leq k-1$.  This assumption is implicitly used as a fact in all the theorems and lemmas. \emph{Therefore, the worst-case response time or the schedulability of task $\tau_k$ has to be verified from $k=1,2,\ldots, n$. Here we only focus on the analysis of a certain task $\tau_k$, under the assumption that we have already validated that $R_i
\leq D_i \leq T_i, \forall \tau_i \mid 1 \leq i \leq k-1$ (by using any method in this section or Section~\ref{sec:existing-analyses}). } 
Our key result in
this paper are the two following theorems:

\begin{theorem}
   \label{theorem:general-framework}
   Suppose that all tasks $\tau_\ell \mid 1 \leq \ell \leq k$ are schedulable, (i.e., $R_\ell \leq T_\ell$). Then, for any arbitrary vector assignment $\vec{x} = (x_1, x_2, \ldots, x_{k-1})$, in which $x_i$ is either $0$ or $1$, the worst-case response time $R_k$ of $\tau_k$ is upper bounded by the minimum $t$ larger than $0$ such that 
   {\small \begin{equation} \label{eq:TDA-suspension-tighter0} 
       C_k + S_k + \sum_{i=1}^{k-1}\ceiling{\frac{t+Q_i^{\vec{x}}+(1-x_i)(R_i-C_i)}{T_i}} C_i \leq t
     \end{equation}}where $Q_i^{\vec{x}} \equals \sum_{j=i}^{k-1} (S_j \times x_j)$.
 \end{theorem} 
\begin{theorem}
   \label{theorem:general-framework-not-feasible}
   Suppose that $\tau_k$ is not schedulable (i.e., $R_k > T_k$). For any arbitrary vector assignment
   $\vec{x} = (x_1, x_2, \ldots, x_{k-1})$, in which $x_i$ is either
   $0$ or $1$,  we have $\forall t | 0 < t \leq T_k$,
   {\small \begin{equation*}
       C_k + S_k + \sum_{i=1}^{k-1}\ceiling{\frac{t+Q_i^{\vec{x}}+(1-x_i)(R_i-C_i)}{T_i}} C_i > t
     \end{equation*}}where $Q_i^{\vec{x}} \equals \sum_{j=i}^{k-1} (S_j \times x_j)$.
 \end{theorem} 
 %We will explain the resulting properties from
 %Theorem~\ref{theorem:general-framework} first, by leaving the proof
 %to Section~\ref{sec:proof-th1} since it is pretty long.   
By Theorems~\ref{theorem:general-framework} and~\ref{theorem:general-framework-not-feasible}, we can directly derive the
following schedulability test.

 \begin{Corollary}
   \label{corollary:general-framework}
   If $\forall \tau_i \mid 1 \leq i < k, ~R_i \leq T_i$, and if there is a vector $\vec{x} = (x_1, x_2, \ldots,
   x_{k-1})$ with $x_i \in \{ 0, 1\}$, such that 
   {\small \begin{align} 
   \label{eq:TDA-suspension-tighter} 
       & \exists t | 0 < t \leq D_k,  \nonumber \\
       & C_k + S_k + \sum_{i=1}^{k-1}\ceiling{\frac{t+Q_i^{\vec{x}}+(1-x_i)(R_i-C_i)}{T_i}} C_i \leq t
     \end{align}}
     where $Q_i^{\vec{x}} \equals \sum_{j=i}^{k-1} (S_j \times x_j)$, then the constrained-deadline task $\tau_k$ is schedulable under fixed-priority.
 \end{Corollary}
 \begin{proof}
Let $t^*$ be the first positive value of $t$ respecting Eq. \eqref{eq:TDA-suspension-tighter}. By the assumptions stated in the claim, $t^*$ exists and $t^*$ is smaller than or equal to the deadline $D_k$. By Theorems~\ref{theorem:general-framework} and~\ref{theorem:general-framework-not-feasible}, $t^*$ exists and is smaller than or equal to $D_k$ only if $\tau_k$ is schedulable.
 \end{proof}

 
  The proof of correctness of Theorems~\ref{theorem:general-framework} and~\ref{theorem:general-framework-not-feasible}, and hence Corollary~\ref{corollary:general-framework} is provided in Section~\ref{sec:proof-th1}. Moreover, we will later prove in Section~\ref{sec:dominance}, that Corollary~\ref{corollary:general-framework} in fact dominates all the analyses discussed in Section~\ref{sec:existing-analyses}.
 
 We now use the same example as in Section~\ref{sec:rationale}, to demonstrate how
Corollary~\ref{corollary:general-framework} can be applied.
 
 \begin{example}
 \label{ex:general_framework}
  Consider the same three tasks used in Examples~\ref{ex:rationale_1} and~\ref{ex:rationale_2}, i.e., $\tau_1 = (4, 5, 10, 10)$, $\tau_2 =(6, 1, 19, 19)$ and $\tau_3 = (4, 0, 50, 50)$. By the analysis in Example~\ref{ex:rationale_1}, $R_1$ is upper bounded by $9$ and $R_2$ is upper bounded by $15$. Let us assume $R_1=9$ and $R_2=15$ in the rest of this example. There are four possible vector assignments $\vec{x}$ when considering the schedulability of task $\tau_3$ with Corollary~\ref{corollary:general-framework}.
% \noindent\textbf{Case 1.} $\vec{x} = (0 , 0)$: In this case, Theorem~\ref{theorem:general-framework} states that $R_k$ is upper bounded by the minimum $t$ under $0 < t \leq D_3$ such that $4+ \ceiling{\frac{t+5}{10}} 4 + \ceiling{\frac{t+9}{19}} 6 \leq t$. Note that this equation is identical to the schedulability test discussed in Section~\ref{sec:jitter}, and hence, as shown in Example~\ref{ex:rationale_1}, we get that $R_k \leq 42$.
%
% \noindent\textbf{Case 2.} $\vec{x} = (0 , 1)$:
%  In this case, Theorem~\ref{theorem:general-framework} states that $R_k$ is upper bounded by  the minimum $t$
% under $0 < t \leq D_3$ that satisfies $4+ \ceiling{\frac{t+6}{10}} 4 + \ceiling{\frac{t+1}{19}} 6 \leq t$. As a solution, we get that $R_k \leq 32$. % due to $4+ \ceiling{\frac{32+7}{10}} 4 + \ceiling{\frac{32+1}{19}} 6=32$.
%       
% \noindent\textbf{Case 3.} $\vec{x} = (1 , 0)$:
%  In this case, we look for the minimum $t$ such that $4+ \ceiling{\frac{t+5}{10}}\cdot 4 + \ceiling{\frac{t+9}{19}}\cdot 6 \leq t$. Hence, we get $R_k \leq 42$.
%
% \noindent\textbf{Case 4.} $\vec{x} = (1 , 1)$:
%  In this case, Theorem~\ref{theorem:general-framework} states that $R_k$ is upper bounded by  the minimum $t$
% under $0 < t \leq T_3$ such that $
%        4+ \ceiling{\frac{t+6}{10}}\cdot 4 + \ceiling{\frac{t+1}{19}}\cdot 6 \leq t$ leading to $R_k \leq 32$.% due to $4+ \ceiling{\frac{32+6}{10}} 4 + \ceiling{\frac{32+1}{19}} 6=32$.
%
% Among the above four cases, the tests in Cases 2 and 4 are the tightest. Therefore, by
% Corollary~\ref{corollary:general-framework}, $\tau_3$ is
% schedulable under fixed-priority.
%
The corresponding procedure to use these four vector assignments can be found in Table~\ref{tab:example3-calculation}. 
Among the above four cases, the tests in Cases 2 and 4 are the
tightest. 
% Therefore, by
%  Corollary~\ref{corollary:general-framework}, $\tau_3$ is
%  schedulable under fixed-priority.
\hfill\myendproof
 \end{example}

Note also that the upper bound on $R_3$ computed in
Example~\ref{ex:general_framework}, is lower than the estimated worst-case response time obtained in Example~\ref{ex:rationale_2}. The response time analysis presented in Corollary~\ref{corollary:general-framework} is therefore tighter than the simple combination of existing analysis techniques as proposed in Example~\ref{ex:rationale_2}.





%%% Local Variables:
%%% mode: latex
%%% TeX-master: "master.tex"
%%% End:
