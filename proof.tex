
\section{Proof of Liu's Analysis}  

This section provides the proof to support the correctness of the test in Eq. \eqref{eq:TDA-suspension}. First, it should be easy to see that we can convert the suspension time of task $\tau_k$ into computation. This has been done by many researchers, e.g., the proof in Lemma 3 in the paper by Liu and Chen \cite{Liu_2014}. The remaining part is to show that the additional interference due to self-suspension from a higher-priority task $\tau_i$ is at most $min(C_i, S_i)$. The interference to be at most $C_i$ has been provided in the literature as well, e.g., \cite{Rajkumar_1990}\cite{Liu_2014}. However, the argument about blocking task $\tau_k$ due to a higher-priority task $\tau_i$ by at most $S_i$ amount of time is not very clear. 

We can prove the correctness of Eq. \eqref{eq:TDA-suspension} by using a similar proof of the critical instant theorem of the ordinary sporadic task system. From the above discussions, we can greedily convert the suspension time of task $\tau_k$ to its computation time. For notational brevity, let $C_k'$ be $C_k + S_k$. We call this converted version of task $\tau_k$ as task $\tau_k'$. Our analysis is also based on a very simple observation as follows:
\begin{lemma}
\label{lemma:remove-lower-priority}
  For a schedule, based on preemptive fixed-priority scheduling, removing a lower-priority job arrived at time $t$ does not change the schedule for executing the higher-priority jobs after time $t$.
\end{lemma}
\begin{proof}
  This is due to the preemptive scheduling. The removal of the lower-priority job has no impact at all on the higher-priority jobs.
\end{proof}
\begin{lemma}
\label{lemma:remove-same-task}
  For a schedule, based on preemptive fixed-priority scheduling, if the worst-case response time of task $\tau_i$ is no more than its period $T_i$, removing a job of task $\tau_i$ arriving at time $t$ does not change the schedule for the remaining jobs of task $\tau_i$.
\end{lemma}
\begin{proof}
  The removal of the job of task $\tau_i$ has no impact on the higher-priority jobs as in Lemma \ref{lemma:remove-lower-priority}. Since the worst-case response time of task $\tau_i$ is no more than the period, the execution of the other jobs of task $\tau_i$ is also not affected by the removal of the job.
\end{proof}

Consider the following schedule:
\begin{itemize}
\item Task $\tau_k'$ releases a job at time $t$.
\item A higher priority task $\tau_i$ in ${\bf T}_1$ releases a job at time $t$, in which the execution time of this job is $C_i+S_i$.
\item A higher priority task $\tau_i$ in ${\bf T}_2$ releases a job at time $t$, in which the execution time of this job is $2C_i$.
\item All the subsequent higher-priority jobs are released as early as possible by respecting to their periods. All these jobs are executed with their worst-case execution times and do not suspend themselves at all.
\end{itemize}

Let the response time of the job of task $\tau_k'$ released at time $t$ in the above schedule be $R_k'$. For brevity, the above schedule is called \emph{bursty interference schedule} for the rest of the report. The following lemma shows that $R_k'$ is a safe upper bound of the original self-suspending task system. 

\begin{lemma}
\label{lemma:critical}
 $R_k'$ defined in the above bursty interference schedule is a safe upper bound of the worst-case response time of task $\tau_k'$ in the self-suspending task system if its worst-case response time is no more than $T_k$.
\end{lemma}
\begin{proof}
According to the above definitions, we only need to show that $R_k'$ is a safe upper bound of the worst-case response time of task $\tau_k'$ by converting the suspension time of task $\tau_k$ as computation. We consider a given schedule in the task system with $\tau_1, \tau_2, \ldots, \tau_k', \tau_{k1+}, \ldots$. Since we consider fixed-priority preemptive scheduling, we can safely remove all the lower priority tasks without changing any execution pattern of the higher-priority tasks by Lemma \ref{lemma:remove-lower-priority}. Moreover, since we assume that the worst-case response time of task $\tau_k'$ is no more than $T_k$, there is no impact on the schedule of a job of task $\tau_k'$ if all the other jobs of task $\tau_k'$ are removed, as shown in Lemma \ref{lemma:remove-same-task}. 


Therefore, for the rest of the proof, we only have to analyze the response time of a job of task $\tau_k'$ released at time $z$, in which all the other jobs of task $\tau_k'$ and other lower priority jobs are removed. Suppose that the job has a response time $\rho$ in the above schedule. This means that the system is busy for executing the higher-priority tasks or the job of task $\tau_k'$ from $z$ to $z+\rho$. In the above schedule, let $t_{k}$ be the latest moment before $t_k$ such that the processor does not run any job. That is, from $t_k$ to $z$, certain higher-priority tasks are executed by the processor. Apparently, we can change the release time of the job of task $\tau_k'$ to $t_k$. The response time of the job becomes $\rho+z-t_k \geq \rho$. 

Up to here, the proof is very similar to the proof of the critical instant theorem of the ordinary sporadic real-time task systems. However, for self-suspending task systems, we need to consider that a job of task $\tau_i$ suspends itself before $t_k$ and resumes after $t_k$.  Fortunately, each higher-priority task only has such a job due to the assumption that the higher-priority tasks can finish before their periods. 

However, analyzing the workload of such \emph{carry-in} jobs due to self-suspension is non-trivial. One can conclude that each job of task $\tau_i$ has execution time up to $C_i$. This is fine with $S_i \geq C_i$. If $S_i < C_i$, we explain how to further extend the analysis window further iteratively. 


In each iteration, we will define $t_j$ as the \emph{release time} of the first job of task $\tau_j$, starting from $j=k-1, k-2, \ldots, 1$, in the revised schedule. Let $y$ be the release time of the job of task $\tau_j$ that has no finished at time $t_{j+1}$. There are a few cases:
\begin{itemize}
\item There is no such a job of task $\tau_j$: Removing all the jobs of task $\tau_j$ arriving before $t_{j+1}$ has no impact on the schedule of the higher-priority jobs executed after $t_{j+1}$ by Lemma \ref{lemma:remove-lower-priority}. Therefore, we simply set $t_j$ to $t_{j+1}$ and remove all the jobs of task $\tau_j$ arriving before $t_{j+1}$ in the schedule
\item There is such a job of task $\tau_j$ with $y < t_{j+1}$:  Removing all the jobs of task $\tau_j$ arriving before $y$ has no impact on the schedule of the higher-priority jobs executed after $t_{j+1}$ by Lemma \ref{lemma:remove-lower-priority}. Therefore, we remove all the jobs of task $\tau_j$ arriving before $y$ in the schedule. There are two subcases:
\begin{itemize}
\item If task $\tau_j$ is in ${\bf T}_1$, i.e., $S_j < C_j$: For such a case, we set $t_{j}$ to $y$. Moreover, we also know that the maximum idle time of the system is at most $S_j$ from $y$ to $t_{j+1}$
\item If task $\tau_j$ is in ${\bf T}_2$, i.e., $S_j \geq C_j$: For such a case, we set $t_{j}$ to $t_{j+1}$. Moreover, we also know that the maximum remaining execution time for the job of arriving before $t_{j}$ is at most $C_j$.
\end{itemize}
\end{itemize}
 
\end{proof}
  
  

  
  
  
  
  
  
  
  
  
  
  
  
  
  
  
  
  